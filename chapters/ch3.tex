% pdf page 25
\caput{Quod Divinum Domini in Caelo, sit Amor in Ipsum et Charitas Erga Proximum.}

\begin{topic}{13}
    Divinum a Domino procedens vocatur in caelo Divinum Verum, ex causa de qua in sequentibus.
    Divinum hoc verum influit in caelum a Domino ex Divino amore Ipsius.
    Divinus amor et inde Divinum verum, se habent comparative sicut ignis solis et lux inde in mundo, amor sicut ignis
    solis, et verum inde sicut lux e sole: ex correspondentia etiam ignis significat amorem, et lux verum inde
    procedens.\footnote{Quod ``ignis'' in Verbo significet amorem utroque sensu (n. 934, 4906, 5215).}
    Inde constare potest, quale est Divinum verum ex Divino amore Domini procedens, quod sit in sua essentia Divinum
    bonum conjunctum Divino vero; et quia conjunctum est, vivificat omnia caeli, sicut calor solis conjunctus luci in
    mundo fructificat omnia telluris, ut fit tempore veris et aestatis; aliter quando calor non conjunctus est luci, ita
    quando lux frigida est, tunc torpent omnia et jacent exstincta.
    Divinum illud bonum, quod comparatum est calori, est bonum amoris apud angelos; ac Divinum verum, quod comparatum
    est luci, est per quod et ex quo bonum amoris.
\end{topic}

\begin{topic}{14}
    Quod Divinum in caelo, quod facit illud, sit amor, est quia amor est conjunctio spiritualis: conjungit ille angelos
    Domino, et conjungit illos inter se mutuo; ac ita conjungit, ut omnes sint sicut unum in conspectu Domini.
    Praeterea amor est ipsum Esse vitae cuique; quare ex illo est vita angelo, et quoque est vita homini.
    Quod ex amore sit intimum vitale hominis, quisque scire potest qui expendit; ex praesentia enim ejus calescit, ex
    absentia ejus frigescit, et ex privatione ejus emoritur.\footnote{Quod amor sit ignis vitae, et quod ipsa vita
    actualiter inde sit (n. 4906, 5071, 6032, 6314).}
    Sed sciendum est, quod talis vita cuique sit, qualis ei amor.
\end{topic}

\begin{topic}{15}
    Sunt bini amores distincti in caelo, amor in Dominum et amor erga proximum; in intimo seu tertio caelo est amor in
    Dominum, et in secundo seu medio caelo est amor erga proximum: uterque procedit a Domino, ac uterque facit caelum.
    Quomodo bini amores se distinguunt, et quomodo se conjungunt, patet in manifesta luce in caelo, at non nisi quam
    obscure in mundo.
    In caelo per amare Dominum non intelligitur amare Ipsum quoad personam, sed amare bonum quod ab Ipso, et amare bonum
    est velle et facere bonum ex amore; et per amare proximum non intelligitur amare socium quoad personam, sed amare
    verum quod ex Verbo, et amare verum est velle et facere verum: inde patet, quod bini illi amores se distinguant
    sicut bonum et verum, et quod se conjungant sicut bonum cum vero.\footnote{Quod amare Dominum et proximum sit vivere
    secundum praecepta Domini (n. 10143, 10153, 10310, 10578, 10648).}
    Sed haec aegre cadunt in ideam hominis, qui non scit quid amor, quid bonum, et quid proximus.\footnote{Quod amare
    proximum non sit amare personam, sed id quod est apud illum ex quo ille, ita verum et bonum (n. 5025[? 5028],
    10336).

    Qui amant personam, et non quod est apud illum ex quo ille, quod ament aeque malum ac bonum (n. 3820).

    Quod charitas sit velle vera et affici veris propter vera (n. 3876, 3877).

    Quod charitas erga proximum sit facere bonum, justum et rectum in omni opere et in omni functione (n. 8120, 8121,
    8122).}
\end{topic}

\begin{topic}{16}
    Locutus sum aliquoties cum angelis de hac re; qui dixerunt, quod mirentur quod homines ecclesiae non sciant, quod
    amare Dominum at amare proximum sit amare bonum et verum, et ex velle facere illa; cum tamen scire possint, quod
    quisque testetur amorem per velle et facere quae alter vult, et quod sic ametur vicissim et conjungatur ipsi, et non
    per quod amet illum, et usque non voluntatem illius facit, quod in se est non amare: et quoque quod possint scire,
    quod bonum procedens a Domino sit similitudo Ipsius, quoniam Ipse est in illo; est quod illi fiant similitudines
    Ipsius, et conjungantur Ipsi, qui bonum et verum vitae suae faciunt, per velle et facere; velle etiam est amare
    facere.
    Quod ita sit, etiam Dominus in Verbo docet, dicendo,
    \begin{quote}
        ``Qui habet praecepta mea, et facit illa, ille est qui amat Me; ...et Ego amabo illum, et mansionem apud illum
        faciam'' (\emph{Joh.} xiv. 21, 23).
    \end{quote}
    Et alibi,
    \begin{quote}
        ``Si mandata mea feceritis, manebitis in amore meo'' (\emph{Joh.} xv. 10, 12).
    \end{quote}
\end{topic}

\begin{topic}{17}
    Quod Divinum a Domino procedens, quod afficit angelos, et facit caelum, sit amor, testatur omnis experientia in
    caelo; omnes enim, qui ibi, sunt formae amoris et charitatis, apparent in pulchritudine ineffabili, ac amor elucet
    ex facie illorum, ex loquela, et ex singulis vitae illorum.\footnote{Quod angeli sint formae amoris [et] charitatis
    (n. 3804, 4735, 4797, 4985, 5199, 5530, 9879, 10177).}
    Praeterea sunt sphaerae vitae spirituales, quae procedunt ex unoquoque angelo et ex unoquoque spiritu, et
    circumfundunt illos, per quas noscuntur quandoque ad multam distantiam, quales sunt quoad affectiones quae amoris;
    nam sphaerae illae effluunt ex vita affectionis et inde cogitationis, seu ex vita amoris et inde fidei cujusvis.
    Sphaerae ab angelis prodeuntes tam plenae sunt amore, ut afficiant intima vitae illorum apud quos sunt; perceptae
    aliquoties a me sunt, ac ita affecerunt.\footnote{Quod sphaera spiritualis, quae est sphaera vitae, effluat et
    exundet ex unoquovis homine, spiritu et angelo, et circumstipet illos (n. 4464, 5179, 7454, 8630).
    Quod effluat ex vita affectionis et inde cogitationis illorum (n. 2489, 4464, 6206).}
    Quod amor sit a quo angeli suam vitam habent, inde etiam patuit, quod unusquisque in altera vita se vertat secundum
    amorem suum; qui in amore in Dominum sunt et in amore erga proximum, se vertunt constanter ad Dominum; qui autem in
    amore sui sunt, se vertunt constanter retro a Domino: hoc fit in omni versura corporis eorum; nam in altera vita
    spatia se habent secundum status interiorum eorum, similiter plagae, quae ibi non determinatae sunt sicut in mundo,
    sed determinantur secundum aspectum faciei illorum: verum non sunt angeli qui se vertunt ad Dominum, sed Dominus qui
    vertit illos ad Se, qui amant facere illa quae ab Ispo.\footnote{Quod spiritus et angeli se convertant constanter ad
    suos amores, et qui in caelis constasnter ad Dominum (n. 10130, 10189, 10420, 10702).

    Quod plagae in altera vita cuique sint secundum aspectum faciei, et inde determinentur, aliter ac in mundo (n.
    10130, 10189, 10420, 10702).}
    Sed de his plura in sequentibus, ubi de Plagis in altera vita.
\end{topic}

\begin{topic}{18}
    Quod Divinum Domini in caelo sit amor, est quia amor est receptaculum omnium caeli, quae sunt pax, intelligentia,
    sapientia, ac felicitas; amor enim recipit omnia et singula quae sibi conveniunt, desiderat illa, inquirit illa,
    imbuit illa sicut sua sponte, nam vult continue locupletari et perfici ab illis:\footnote{Quod amori insint
    innumerabilia, et quod amor recipiat ad se omnia quae concordant (n. 2500, 2572, 3078, 3189, 6323, 7490, 7750).}
    quod etiam notum est homini, nam amor apud illum quasi inspicit et haurit ex rebus memoriae ejus omnia quae
    concordant, ac illa colligit, et disponit in se et sub se, in se ut sint sua, ac sub se ut sibi inserviant; cetera
    autem, quae non concordant, rejicit et exterminat.
    Quod amori insit omnis facultas recipiendi vera sibi convenientia, ac desiderium illa sibi conjungendi, patuit etiam
    manifeste ab illis qui in caelum evecti sunt; illi tametsi simplices in mundo fuerunt, usque in sapientiam angeliam
    et in felicia caeli venerunt, cum inter angelos: causa fuit, quia amaverunt bonum et verum propter bonum et verum,
    et implantaverunt illa vitae suae, et per id facultates facti sunt recipiendi caelum cum omni ineffabili ibi.
    Qui autem in amore sui et mundi sunt, illi in nulla facultate recipiendi illa sunt, aversantur illa, rejiciunt
    illa, et ad primum tactum et influxum eorum aufugiunt, et se associant illis in inferno, qui in similibus secum
    amoribus sunt.
    Fuerunt spiritus qui dubitabant quod talia inessent amori caelesti, et desiderabant scire num ita esset; quapropter
    missi sunt in statum amoris caelestis, remotis interea obstantibus, et perlati antrorsum ad distantiam ubi caelum
    angelicum, et inde locuti sunt mecum, dicentes quod interiorem felicitatem percipiant, quam vocibus exprimere
    queunt, dolentes valde quod in prinstinum statum redirent.
    Alii etiam elevati sunt in caelum; et sicut interius seu altius sublati sunt, ita intraverunt in intelligentiam et
    sapientiam, ut percipere possent, quae prius illis incomprehensibilia fuerunt.
    Inde patet, quod amor procedens a Domino sit receptaculum caeli et omnium ibi.
\end{topic}

\begin{topic}{19}
    Quod amor in Dominum et amor erga proximum comprehendant in se omnia vera Divina, constare potest ex illis quae Ipse
    Dominus de binis illis amoribus locutus est, dicendo,
    \begin{quote}
        ``Amabis..Deum tuum ex toto corde tuo et ex tota anima tua...; hoc est praeceptum maximum et primum;
        secundum..quod simile est illi, est, ut ames proximum tuum sicut te ipsum: ex his duobus praeceptis pendent Lex
        et Prophetae'' (\emph{Matth.} xxii. 37-40).
    \end{quote}
    ``Lex et Prophetae'' sunt totum Verbum, ita omne Verum Divinum.
\end{topic}
