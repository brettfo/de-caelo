% pdf page 42
\caput{Quod Unaquaevis Societas sit Caelum in Minore Forma, et Unusquisque Angelus in Minima.}

\begin{topic}{51}
    Quod unaquaevis societas sit caelum in minore forma, et unusquisque angelus in minima, est quia bonum amoris et
    fidei est quod facit caelum, et id bonum est in omni societate caeli, et in omni angelo societatis.
    Nihil refert, quod bonum illud ubivis differat et varium sit; est usque bonum caeli; differentia modo est, quod
    caelum tale sit hic et tale ibi.
    Ideo dicitur, cum quis elevatur in aliquam societatem caeli, quod veniat in caelum; et de illis qui ibi, quod sint
    in caelo, et quisque in suo: hoc norunt omnes qui in altera vita; ideo qui extra aut infra caelum stant, et spectant
    e longinquo ubi coetus angelorum sunt, dicant quod caelum sit ibi, et quoque ibi.
    Se habet hoc comparative sicut cum praefectis, officiariis et ministris in uno palatio regio aut in una aula;
    tametsi habitant seorsim in suis mansionibus aut in suo conclavi, unus supra, alter infra, usque sunt in uno palatio
    aut in una aula, quisque ibi in sua functione ad serviendum regi.
    Inde patet quid intelligitur per Domini verba quod
    \begin{quote}
        In domo Patris sui sint mansiones multae (\emph{Joh.} xiv. 2);
    \end{quote}
    et quid per ``habitacula caeli,'' et per ``caelos caelorum'' apud \emph{Prophetas}.
\end{topic}

\begin{topic}{52}
    Quod unaquaevis societas sit caelum in minore forma, etiam constare potuit ex eo, quod similis forma caelestis sit
    in quavis societate, qualis est in toto caelo: in toto caelo sunt in medio qui praestant reliquis, et circumcirca
    usque ad terminos sunt in ordine decrescente qui minus praestant, ut dictum videatur in articulo praecedente (n.
    43); et quoque ex eo, quod Dominus ducat omnes qui in toto caelo tanquam essent unus angelus; similiter illos qui in
    unaquavis societate: inde apparet quandoque integra societas angelica sicut unum in forma angeli, quod etiam mihi
    datum est a Domino videre.
    Cum etiam Dominus apparet in medio angelorum, tunc non apparet circumstipatus a pluribus, sed ut unus in Forma
    angelica: inde est, quod Dominus in Verbo dicatur Angelus; et quoque quod integra societas; Michael, Gabriel et
    Raphael non sunt nisi quam societates angelicae, quae a functionibus suis ita nominantur.\footnote{Quod Dominus in
    Verbo dicatur Angelus (n. 6280, 6831, 8192, 9303).

    Quod integra societas angelica dicatur angelus; et quod Michael et Raphael sint societates angelicae ex functionibus
    ita dictae (n. 8192).

    Quod societates caeli, et angeli non aliquod nomen habeant, sed quod dignoscantur ex quali boni, et ex idea de ilio
    (n. 1705, 1754).}
\end{topic}

\begin{topic}{53}
    Sicut integra societas est caelum in minore forma, ita quoque est angelus caelum in minima; nam caelum non est extra
    angelum, sed intra illum; interiora enim ejus, quae sunt mentis ejus, disposita sunt in formarti caeli, ita ad
    receptionem omnium caeli quae extra illum sunt; recipit etiam illa secundum quale boni, quod est in illo ex Domino;
    inde est angelus quoque caelum.
\end{topic}

\begin{topic}{54}
    Nusquam dici potest, quod caelum sit extra aliquem, sed intra; nam omnis angelus secundum caelum quod intra illum
    est, recipit caelum quod extra illum est.
    Inde patet, quantum fallitur qui credit quod venire in caelum, sit solum elevari inter angelos, qualiscunque sit
    quoad vitam interiorem suam; ita quod caelum detur cuique ex immediata misericordia;\footnote{Quod caelum non
    donetur ex immediata misericordia, sed secundum vitam, et quod omne vitae per quod homo a Domino ducitur ad caelum,
    sit ex misericordia, et quod id intelligatur (n. 5057, 10659).

    Si caelum donaretur ex immediata misericordia, quod omnibus donaretur (n. 2401).

    De quibusdam malis e caelo dejectis qui crediderunt caelum dari cuivis ex immediata misericordia (n. 4726[? 4226]).}
    cum tamen nisi caelum sit intra aliquem, nihil caeli quod extra est, influit et recipitur.
    Sunt multi spiritus, qui in tali opinione sunt, ac ideo quoque propter fidem suam, in caelum evecti sunt; sed cum
    ibi erant, quia interior vita eorum contraria erat vitae in qua angeli, coeperunt quoad intellectualia sua
    occaecari, usque ut facti sint sicut fatui, et quoad voluntaria sua cruciari usque ut gererent se sicut dementes:
    verbo, qui male vivunt, et in caelum veniunt, trahunt ibi animam et torquentur comparative sicut pisces extra aquas
    in atmosphaera; et sicut animalia in antliis pneumaticis in aethere extracto aere.
    Inde constare potest, quod caelum sit intra et non extra aliquem.\footnote{Quod caelum sit in homine (n. 3884).}
\end{topic}

\begin{topic}{55}
    Quia omnes recipiunt caelum quod extra illos est secundum quale caeli quod intra illos, ideo similiter recipiunt
    Dominum quoniam Divinum Domini facit caelum: inde est, quod cum Dominus Se praesentem sistit in aliqua societate,
    ibi appareat secundum quale boni in quo est societas, ita non similiter in una societate ut in altera: non quod
    dissimilitudo llla sit in Domino, sed in illis qui vident Ipsum ex suo bono, ita secundum illud; afficiuntur etiam
    Ipso viso secundum quale sui amoris; qui intime amant Ipsum, intime afficiuntur, qui minus amant minus afficiuntur;
    mali qui extra caelum sunt, ad praesentiam Ipsius cruciantur.
    Cum Dominus apparet in aliqua societate, apparet ibi ut Angelus; sed dignoscitur ab aliis per Divinum quod
    translucet.
\end{topic}

\begin{topic}{56}
    Caelum etiam est, ubi Dominus agnoscitur, creditur, et amatur; varietas cultus Ipsius ex varietate boni in societate
    una et altera, non fert damnum, sed fert emolumentum; nam perfectio caeli inde est.
    Quod perfectio caeli inde sit, aegre ad captum explicari potest, nisi in opem adhibeantur voces in litterato orbe
    sollennes et usitatae, et per illas exponatur quomodo unum quod perfectum ex variis formatur: omne unum ex variis
    existit, nam unum, quod non ex variis, non est aliquid, non habet formam, et ideo non habet quale: cum autem unum
    existit ex variis, et varia sunt in forma perfecta, in qua quodlibet adjungit se alteri ut amicum consentiens in
    serie, tunc habet quale perfectum: caelum etiam est unum ex variis in perfectissimam formam ordinatis; nam forma
    caelestis est omnium formarum perfectissima.
    Quod omnis perfectio inde sit, patet ab omni pulchritudine, amoenitate, et jucunditate, quae afficiunt tam sensus
    quam animos; illae enim non aliunde existunt et fluunt quam ex concentu et harmonia plurium concordantium et
    consentientium, sive ea in ordine coexistant, sive in ordine consequantur, et non ex uno absque pluribus: inde
    dicitur quod varietas delectet, et scitur quod delectatio sit secundum quale ejus.
    Ex his sicut in speculo videri potest, unde perfectio ex variis sit, etiam in caelo; nam ex illis quae in mundo
    naturali existunt, sicut in speculo videri possunt quae in mundo spirituali.\footnote{Quod omne unum sit ex harmonia
    et consensu plurium, et quod alioqui non sit ei quale (n. 457).

Quod inde universum caelum sit unum (n. 457).

Et hoc ex eo quod omnes ibi spectent unum finem, qui est Dominus (n. 9828).}
\end{topic}

\begin{topic}{57}
    De ecclesia simile dici potest quod de caelo, nam ecclesia est caelum Domini in terris.
    Sunt illae quoque plures, et tamen unaquaevis vocatur ecclesia, et quoque est ecclesia, quantum bonum amoris et
    fidei ibi regnat.
    Dominus etiam ibi ex variis unum facit, ita ex pluribus ecclesiis unam.\footnote{Si bonum foret character et
    essentiale ecclesiae, et non verum absque bono, quod ecclesia foret una (n. 1285, 1316, 2982, 3267, 3445, 3451,
    3452).

    Quod etiam omnes ecclesiae faciant unam ecclesiam coram Domino ex bono (n. 7395[? 7396], 9276).}
    Simile quoque dici potest de homine ecclesiae in particulari, quod de ecclesia in communi; quod nempe ecclesia sit
    intra hominem, et non extra illum, et quod quilibet homo sit ecclesia, in quo Dominus est praesens in bono amoris et
    fidei.\footnote{Quod ecclesia sit in homine, et non extra illum, et quod ecclesia in communi sit ab hominibus, in
    quibus ecclesia (n. 3884).}
    Simile etiam dici potest de homine in quo ecclesia, quod de angelo in quo caelum, quod sit ecclesia in minima forma,
    sicut angelus est caelum in minima forma: et adhuc magis, quod homo in quo ecclesia, aeque ac angelus, sit caelum;
    nam homo creatus est ut in caelum veniat, et fiat angelus; quapropter ille, cui bonum est a Domino, est angelus
    homo.\footnote{Quod homo qui ecclesia sit caelum in minima forma ad imaginem maximi, quia interiora ejus quae mentis
    disposita sunt ad formam caeli, et inde ad receptionem omnium caeli (n. 911, 1900, 1982[? 1928], 3624-3614, 3634,
    3884, 4041, 4279, 4523, 4524, 4625, 6013, 6057, 9279, 9632).}
    Memorare licet, quid homo commune habet cum angelo, et quid prae angelis: Homo commune habet cum angelo, quod
    interiora ejus aeque formata sint ad imaginem caeli, et quoque quod fiat imago caeli, quantum in bono amoris et
    fidei est: homo prae angelis habet, quod exteriora ejus formata sint ad imaginem mundi; et quod quantum in bono est,
    mundus apud illum subordinatur caelo, et serviat caelo;\footnote{Quod homini sit internum et externum, et quod
    internum ejus a creatione formatum sit ad imaginem caeli, et quod externum ejus ad imaginem mundi, et quod ideo homo
    ab antiquis dictus sit microcosmus (n. 4523, 4524, 5368[? 3628, 5115], 6013, 6057, 9279, 9706, 10156, 10472).

    Quod ideo homo ita creatus sit, ut mundus apud illum serviat caelo; quod etiam fit apud bonos, at quod inversum sit
    apud malos, ubi caelum servit mundo (n. 9283, 9278).} et quod tunc Dominus praesens sit apud illum in utroque sicut
    in suo caelo; est enim in suo ordine Divino utrobivis, nam Deus est ordo.\footnote{Quod Dominus sit ordo, quoniam
    Divinum bonum et verum, quae procedunt a Domino, faciunt ordinem (n. 1728, 1919, 2201[? 2011], 2258, 5110, 5703,
    8988, 10336, 10619).

    Quod vera Divina sint leges ordinis (n. 2247, 7995).

    Quod quantum homo secundum ordinem vivit, ita quantum in bono secundum vera Divina, tantum sit homo, et in illo
    ecclesia et caelum (n. 4839, 6605, 8067[? 8513]).}
\end{topic}

\begin{topic}{58}
    Ultimo memorandum est, quod qui caelum in se habet, non modo habeat caelum in suis maximis seu communibus, sed etiam
    in suis minimis seu singularibus; et quod minima ibi, in imagine referant maxima.
    Hoc venit ex eo, quod unusquisque sit suus amor, et talis qualis ejus amor regnans; quod regnat hoc influit in
    singula, et disponit illa, et ubivis inducit similitudinem sui:\footnote{Quod amor regnans seu dominans apud
    unumquemvis sit in omnibus et singulis ejus vitae, ita in omnibus et singulis ejus cogitationis et voluntatis (n.
    6159, 7648, 8067, 8853).

    Quod homo talis sit quale ejus vitae regnans (n. 918, 1040, 1568, 1571[? \emph{dele}], 3570, 6571, 6934[? 6935],
    6938, [8853,] 8854, 8856, 8857, [8858,] 10076, 10109, 10110, 10284).

    Quod amor et fides, cum regnant sint in singulis vitae hominis, tametsi id nescit (n. 8854, 8864, 8865).} in caelis
    est amor in Dominum regnans, quia Dominus ibi supra omnia amatur; inde Dominus est ibi omne in omnibus, influit in
    omnes et singulos, disponit illos, et induit similitudinem sui, et facit ut caelum sit ubi Ille: inde angelus est
    caelum in minima forma, societas in majore, et omnes societates simul sumptae in maxima.
    Quod Divinum Domini faciat caelum, et quod sit omne in omnibus videatur supra (n. 7-12).
\end{topic}
