% pdf page 48
\caput{Quod Universum Caelum in Uno Complexu Referat Unum Hominem.}

\begin{topic}{59}
    Quod caelum in toto complexu referat unum hominem, est arcanum nondum in mundo notum; in caelis autem est
    notissimum; id scire, ac specifica et singularia de eo, est praecipuum intelligentiae angelorum ibi; inde etiam
    pendent plura, quae absque illo ut suo communi principio non distincte et clare intrarent in ideas mentis eorum.
    Quia sciunt, quod omnes caeli una cum societatibus eorum referant unum hominem, ideo etiam caelum vocant
    \emph{Maximum et Divinum Hominem};\footnote{Quod caelum in toto complexu appareat in forma sicut Homo, et quod
    caelum inde dicatur Maximus Homo (n. 2996, 2998, 3624-3649, 3636-3643. 3741-3745, 4625).} Divinum ex eo, quod
    Divinum Domini faciat caelum (vide supra, n. 7-12).
\end{topic}

\begin{topic}{60}
    Quod caelestia et spiritualia in illam formam et in illam imaginem ordinata et conjuncta sint, non percipere possunt
    qui de spiritualibus et caelestibus non justam ideam habent; cogitant illi, quod terrestria et materialia, quae
    componunt ultimum hominis, faciant illum, et quod absque illis homo non sit homo: sed sciant, quod homo non sit homo
    ex illis, sed ex eo quod intelligere possit verum et velle bonum; haec sunt spiritualia et caelestia, quae faciunt
    hominem.
    Novit etiam homo, quod quisque talis homo sit, qualis est quoad intellectum et voluntatem; et quoque nosse potest,
    quod terrestre ejus corpus sit formatum ad serviendum illis in mundo, et ad conformiter praestandum illis usus in
    ultima naturae sphaera; ideo etiam corpus nihil agit ex se, sed agitur prorsus obsequiose ad nutus intellectus et
    voluntatis, usque adeo ut quicquid homo cogitat, loquatur lingua et ore, et quicquid vult, faciat corpore et
    membris, sic ut intellectus et voluntas sit faciens, et nihil corpus a se: inde patet, quod intellectualia et
    voluntaria faciant hominem, et quod illa sint in simili forma, quia illa agunt in singularissima corporis sicut
    internum in externum; homo itaque ex illis vocatur internus et spiritualis homo.
    Talis homo in maxima et perfectissima forma est caelum.
\end{topic}

\begin{topic}{61}
    Talis idea est angelorum de nomine; quapropter nusquam attendunt ad illa quae homo corpore facit, sed ad voluntatem
    ex qua corpus facit: hanc vocant ipsum hominem, et intellectum quantum ille cum voluntate unum agit.\footnote{Quod
    voluntas hominis sit ipsum Esse vitae illius, et quod intellectus sit Existere vitae inde (n. 3619, 5002, 9282).

    Quod vita voluntatis sit principalis vita hominis, et quod vita intellectus procedat inde (n. 585, 590, 3619, 7342,
    8885, 9282, 10076, 10109, 10110).

    Quod homo sit homo ex voluntate et inde intellectu (n. 8911, 9069, 9071, 10076, 10109. 10110).}
\end{topic}

\begin{topic}{62}
    Angeli quidem non vident caelum in toto complexu in tali forma, nam totum caelum non cadit in conspectum alicujus
    angeli, sed vident quandoque dissitas societates, quae ex multis millibus angelorum consistunt, ut unum in tali
    forma; et ex societate ut ex parte concludunt ad commune quod est caelum; nam in perfectissima forma communia se
    habent sicut partes, et partes sicut communia; discrimen modo est sicut inter simile majus et minus.
    Inde dicunt, quod totum caelum tale sit in conspectu Domini, quia Divinum ex intimo et supremo omnia videt.
\end{topic}

\begin{topic}{63}
    Quia caelum tale est, ideo quoque regitur illud a Domino sicut unus homo, et inde sicut unum: notum enim est, quod
    tametsi homo consistit ex innumerabilibus variis, tam in toto quam in parte, in toto ex membris, organis, et
    visceribus, in parte ex seriebus fibrarum, nervorum, et vasorum sanguineorum, ita ex membris intra membra et ex
    partibus intra partes, usque tamen homo, cum agit, sicut unus agit: tale etiam est caelum sub auspicio et ductu
    Domini.
\end{topic}

\begin{topic}{64}
    Quod tot varia in nomine unum agant, est quia nihil quicquam ibi est, quod non aliquid facit ad rem communem, et
    praestat usum; commune praestat usum partibus suis, et partes praestant usum communi, nam commune est ex partibus et
    partes constituunt commune, quare prospiciunt sibi invicem, spectant se mutuo, et conjunguntur in tali forma ut
    omnia et singula se referant ad commune et ejus bonum; inde est, quod unum agant.
    Similes sunt consociationes in caelis; conjunguntur ibi secundum usus in simili forma; quare qui non usum praestant
    communi, ejiciuntur e caelo, quia sunt heterogenea.
    Usum praestare est aliis velle bene propter commune bonum, et usum non praestare est aliis velle bene non propter
    commune bonum sed propter se; hi sunt qui amant se supra omnia, illi autem sunt qui amant Dominum supra omnia: inde
    est, quod illi qui in caelo sunt, unum agant, sed hoc non ex se sed ex Domino, spectant enim Ipsum ut Unicum a Quo,
    ac regnum Ipsius ut commune, cui consulendum.
    Hoc intelligitur per Domini verba,
    \begin{quote}
        ``Quaerite..primo Regnum Dei, et justitiam ejus, et..omnia adjicientur vobis'' (\emph{Matth.} vi. 33).
    \end{quote}
    ``quaerere justitiam ejus'' est bonum ejus.\footnote{Quod justitia in Verbo dicatur de bono, judiciurn de vero; inde
    facere justitiam et judicium est facere bonum et verum (n. 2235, 9857).}
    Qui in mundo amant patriae bonum plus quam suum, et proximi bonum sicut suum, illi sunt qui in altera vita amant et
    quaerunt regnum Domini, nam ibi regnum Domini est loco patriae; et qui amant facere aliis bonum, non propter se sed
    propter bonum, illi amant proximum, nam ibi bonum est proximus:\footnote{Quod Dominus in supremo sensu sit proximus,
    et inde quod amare Dominum sit amare id quod ab Ipso, quia in omni quod ab Ipso est Ipse, ita bonum et verum (n.
    2425, 3419, 6706, 6711, 6819, 6823, 8123).

    Inde quod omne bonum quod a Domino sit proximus, et quod velle et facere id bonum sit amare proximum (n. 5026[?
    5028], 10336).} omnes illi qui tales sunt, in Maximo Homine, hoc est, caelo, sunt.
\end{topic}

\begin{topic}{65}
    Quia totum caelum refert unum hominem, et quoque est Divinus Spiritualis Homo in maxima forma, etiam in effigie,
    ideo distinguitur caelum in membra et partes, sicut homo, et quoque nominantur similiter: sciunt etiam angeli, in
    quo membro una societas est, et in quo altera; et dicunt, quod illa societas sit in membro seu provincia aliqua
    capitis, illa in membro seu provincia aliqua pectoris, illa in membro aut provincia aliqua lumborum, et sic porro.
    In genere, caelum supremum seu tertium format caput usque ad collimi; caelum medium seu secundum format pectus usque
    ad lumbos et genua: caelum ultimum seu primum format pedes usque ad plantas, et quoque brachia usque ad digitos; nam
    brachia et manus sunt ultima hominis, tametsi a latere.
    Inde iterum patet, cur tres caeli sunt.
\end{topic}

\begin{topic}{66}
    Spiritus qui infra caelum sunt mirantur valde, cum audiunt et vident quod caelum sit tam infra quam supra; sunt enim
    in simili fide et opinione, in quali sunt homines in mundo, quod caelum non alibi sit quam supra; non enim sciunt,
    quod situs caelorum sit sicut situs membrorum, organorum, et viscerum in homine, quorum quaedam sunt supra et
    quaedam infra; et quod sit sicut situs partium in unoquovis membro, organo, et viscere, quarum quaedam sunt intra,
    quaedam extra; inde confundunt se de caelo.
\end{topic}

\begin{topic}{67}
    Haec de caelo ut Maximo Homine aliata sunt, quia absque illa cognitione praevia nullatenus capi possunt quae
    sequuntur de caelo; nec potest aliqua distinta idea haberi de forma caeli, de conjunctione Domini cum caelo, de
    conjunctione caeli cum homine, nec de influxu mundi spiritualis in naturalem, et prorsus non aliqua de
    correspondentia, de quibus tamen ordine in nunc sequentibus agendum est; quapropter ad dandum lucem in illis, hoc
    praemissum est.
\end{topic}
