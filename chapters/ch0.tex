% pdf page 17
\chapter*{Praefatio.}

\begin{topic}{1}
    Ubi Dominus de consummatione saeculi, quae est ultimum tempus ecclesiae,\footnote{Consummatio saeculi, quod sit
    ultimum tempus ecclesiae (n. 4535, 10672[? 10622]).} coram discipulis loquitur, ad finem praedictionum de
    successivis statibus ejus quoad amorem et fidem,\footnote{Explicantur quae Dominus de consummatione saeculi, deque
    adventu Ipsius, ita de successiva vastatione ecclesiae et de ultimo judicio, praedixerat apud \emph{Matthaeum,}
    cap., xxiv. et xxv., in initiis ad capita xxvi. ad xxxix.  \emph{Genes.;} et ibi n. 3353-3356, 3486-3489, 3650-3655,
    3751-3759[? 3757], 3897-3901, 4056-4060, 4229-4231, 4332-4335, 4422-4424, 4635-4638, 4661-4664, 4807-4810,
    4954-4959, 5063-5071).} ita dicit:
    \begin{quote}
        ``Statim..post afflictionem dierum istorum sol obscurabitur, et luna non dabit lumen suum, et sellae cadent de
        caelo, et potentiae caelorum commovebuntur.
        Et tunc apparebit signum Filii hominis is caelo; et tunc plangent omnes tribus terrae: et videbunt Filium
        hominis venientem in nubibus caeli cum potentia et gloria multa.
        Et emittet angelos suos cum tuba et voce magna, et congregabunt electos Ipsius a quatuor ventis, ab extremo
        caelorum usque ad extremum illorum'' (\emph{Matth.} xxiv. 29-31).
    \end{quote}
    Qui secundum sensum litterae illa verba intelligunt, non aliter credunt, quam quod omnia illa secundum descriptionem
    in illo sensu postremo tempore, quod vocatur Ultimum Judicium, eventura sint; ita non solum quod sol et luna
    obscurabuntur, et quod stellae cadent de caelo, quodque appariturum signum Domini in caelo, et quod visuri Ipsum in
    nubibus, et simul angelos cum tubis, sed etiam secundum praedictiones alibi, quod totus mundus aspectabilis
    periturus sit, ac postea novum caelum cum nova terra exstiturum: in hac opinione sunt plerique hodie intra
    ecclesiam.
    Sed qui ita credunt, non sciunt arcana quae latent in singulis Verbi; in singulis enim Verbi est sensus internus; in
    quo non naturalia et mundana, qualia sunt quae in sensu litterae, sed spiritualia et calestia, intelliguntur; et hoc
    non modo quoad sensum plurium vocum, sed etiam quoad unamquamvis vocem;\footnote{Quod in omnibus et singulis Verbi
    sit sensus internus seu spiritualis (n. 1143, 1984, 2135, 2333, 2395, 2495, 4442, 9049[? 9048], 9063, 9086).} Verbum
    enim conscriptum est per meras correspondentias,\footnote{Quod Verbum per meras correspondentias conscriptum sit, et
    quod inde omnia et singula ibi significent spiritualia (n. 1404, 1408, 1409, 1540, 1619, 1659, 1709, 1783, 2900,
    9086).} ob finem ut sin singulis sensus internus sit.
    Qualis ille sensus est, constare potest ex omnibus illis quae de eo sensu in \emph{Arcanis Caelestibus} dicta et
    ostensa sunt; quae etiam inde collata videantur in explicatione \emph{De Equo Albo}, de quo in \emph{Apocalypsi}.
    Secundum eundem sensum intelligenda sunt quae Dominus in supra allato loco de adventu suo in nubibus caeli locutus
    est; per ``solem'' ibi, qui obscurabitur, significatur Dominus quoad amorem;\footnote{Quod ``sol'' in Verbo
    significet Dominum quoad amorem, et inde amorem in Dominum (n. 1529, 1837, 2441, 2495, 4060, 4696, 4996, 7083,
    10809).} per ``lunam'' Dominus quoad fidem;\footnote{Quod ``luna'' in Verbo significet Dominum quoad fidem, et inde
    fidem in Dominum (n. 1529, 1530, 2495, 4060, 4996[? 4696], 7083).} per ``stellas'' cognitiones boni et veri, seu
    amoris et fidei;\footnote{Quod ``stellae'' in Verbo significent cognitiones boni et veri (n. 2495, 2849, 4697).} per
    ``signum Filii hominis in caelo,'' apparatio Divini veri; per ``tribus terrae.'' quae plangent, omnia veri et boni,
    seu fidei et amoris;\footnote{Quod ``tribus'' significent omnia vera et bona in complexu, ita omnia fidei et amoris
    (n. 3858, 3926, 4060, 6335).} per ``adventum Domini in nubibus caeli cum potentia et gloria,'' praesentia Ipsius in
    Verbo, et revelatio;\footnote{Quod ``adventus Domini'' sit praesentia Ipsius in Verbo, ac revelatio (n. 3900, 4060).
    } per ``nubes'' significatur Verbi sensus litterae,\footnote{Quod ``nubes'' in Verbo significent Verbum in littera,
    seu sensum litterae ejus (n. 4060, 4391, 5922, 6343, 6752, 8106, 8781, 9430, 10551, 10574).} et per ``gloriam''
    Verbi sensus internus;\footnote{Quod ``gloria'' in Verbo significet Divinum verum quale est in caelo, et quale est
    in sensu interno Verbi (n. 4809, 5292, 5922, 8267, 8427, 9429, 10574).} per ``angelos cum tuba et voce magna''
    significatur caelum unde Divinum verum.\footnote{Quod ``tuba'' seu ``buccina'' significet Divinum verum in caelo, et
    e caelo revelatum (n. 8815, 8823, 8915); [quod hoc] similiter [significetur] per ``vocem'' (n. 6971, 9926).}
    Inde constare potest, quod per illa verba Domini intelligatur, quod in fine ecclesiae, quando amplius non amor et
    inde non fides, Dominus aperturus sit Verbum quoad sensum ejus internum, et quod revelaturus arcana caeli: arcana,
    quae in nunc sequentibus revelantur, sunt de Caelo et de Inferno, et simul de Vita hominis post mortem.
    Homo ecclesiae hodie vix aliquid novit de caelo et de inferno, nec de vita sua post mortem, tametsi omnia descripta
    exstant in Verbo; immo etiam multi, qui intra ecclesiam nati sunt, negant illa, dicentes corde suo, ``Quis inde
    venit et narravit?''
    Ne itaque tale negativum, quod imprimis regnat apud illos qui multa e mundo sapiunt, etiam inficiat et corrumpat
    simplices corde et simplices fide, datum est mihi una esse cum angelis, et loqui cum illis sicut homo cum homine, et
    quoque videre quae in caelis, tum quae in infernis, et hoc per tredecim annos; ita nunc illa ex Visis et Auditis
    describere; sperans sic ignorantiam illustrari, et incredulitatem dissipari.
    Quod hodie immediata talis revelatio existat, est quia illa est quae per adventum Domini iltelligitur.
\end{topic}
