% pdf page 38
\caput{Quod Caeli Consistant ex Innumeris Societatibus.}

\begin{topic}{41}
    Angeli cujusvis caeli non sunt in uno loco simul, sed distincti in societates majores et minores, secundum
    differentias boni amoris et fidei in quo sunt; qui in simili bono sunt, unam societatem formant. Bona in caelis in
    infinita varietate sunt; et unusquisque angelus est sicut suum bonum.\footnote{Quod infinita varietas sit, et
    nusquam aliquid idem cum altero (n. 7236, 9002).

    Quod in caelis etiam infinita varietas sit (n. 684, 690, 3744, 5598, 7236).

    Quod varietates in caelis, quae infinitae, sint varietates boni (n. 3744, 4005, 7236, 7833, 7836, 9002).

    Quod illae varietates existant per vera quae multiplicia, ex quibus bonum cuivis (n. 3470, 3804, 4149, 6917, 7236).

    Quod inde omnes societates in caelis, et unusquisque angelus in societate, a se invicem distincti sint (n. 690,
    3241, 3519, 3804, 3986, 4067, 4149, 4263, 7236, 7833, 7836).

    Sed quod usque omnes unum agant per amorem a Domino (n. 457, 3986).}
\end{topic}

\begin{topic}{42}
    Societates angelicae in caelis etiam distant inter se, sicut differunt bona in genere et in specie; nam distantiae
    in mundo spirituali non ex alia origine sunt, quam ex differentia status interiorum, inde in caelis ex differentia
    statuum amoris; distant multum qui differunt multum, et distant parum, qui differunt parum; similitudo facit ut una
    sint.\footnote{Quod omnes societates caeli constantem situm habeant secundum differentias status vitae, ita secundum
    differentias amoris et fidei (n. 1274, 3638, 3639).

    Mirabilia in altera vita seu in mundo spirituali de distantia, situ, loco, spatio et tempore (n. 1273-1277).}
\end{topic}

\begin{topic}{43}
    Omnes in una societate similiter inter se distincti sunt: qui perfectiores sunt, hoc est, qui praestant bono, ita
    amore, sapientia et intelligentia, in medio sunt; qui minus praestant, circumcirca sunt, ad distantiam secundum
    gradus prout diminuitur perfectio. Se habet hoc, sicut lux e medio decrescens ad peripherias: qui in medio sunt,
    etiam in maxima luce sunt; qui ad peripherias in minore et minore.
\end{topic}

\begin{topic}{44}
    Similes quasi ex se feruntur ad similes; nam sunt cum similibus sicut cum suis, et sicut domi; cum aliis autem sicut
    cum peregrinis, et sicut foris: quando apud similes sunt, etiam in suo libero sunt, et inde in omni jucundo vitae.
\end{topic}

\begin{topic}{45}
    Inde patet, quod bonum consociet omnes in caelis, et quod distinguantur secundum ejus quale: at usque non angeli
    sunt, qui se ita consociant, sed Dominus a quo bonum; Ipse ducit illos, conjungit illos, distinguit illos, et tenet
    illos in libero quantum in bono; ita unumquemvis in vita sui amoris, suae fidei, suae intelligentiae et sapientiae,
    et inde in felicitate.\footnote{Quod omne liberum sit amoris et affecìtonis, quoniam quod homo amat, hoc libere
    facit (n. 2870, 3158, 8907[? 8987], 8990, 9585. 9591).

    Quia liberum est quod est amoris, quod inde sit vita cujusvis et ejus jucundum (n. 2873).

    Quod nihil appareat ut proprium, nisi quod ex libero (n. 2880).

    Quod ipsissimum liberum sit duci a Domino, quia sic ducitur ab amore boni et veri (n. 892, 905, 2872, 2886, 2890,
    2891, 2892, 9096, 9586-9591).}
\end{topic}

\begin{topic}{46}
    Cognoscunt etiam se omnes qui in simili bono sunt, prorsus sicut homines in mundo suos propinquos, suos affines, et
    suos amicos, tametsi illos nusquam prius viderunt; ex causa, quia in altera vita non sunt propinquitates,
    affinitates, et amicitiae aliae quam spirituales, ita quae sunt amoris et fidei.\footnote{Quod omnes proximitates,
    cognitiones, affinitates et quasi consanguinitates in caelo sint ex bono, et secundum ejus convenientias et
    differentias (n. 605[? 685], 917, 1394, 2739, 3612, 3815, 4121).} Hoc mihi aliquoties datum est videre, quando in
    spiritu fui, ita abductus a corpore, et sic in consortio cum angelis: tunc quosdam ex illis vidi sicut notos ab
    infantia, alios vero sicut prorsus non notos; qui visi sicut noti ab infantia, fuerunt qui in simili statu cum statu
    spiritus mei erant; qui autem non noti, in dissimili.
\end{topic}

\begin{topic}{47}
    Omnes qui unam societatem angelicam formant, simili facie sunt in communi, sed non simili in particulari.
    Quomodo similitudines in communi et variationes in particulari se habent, aliquantum comprehendi potest ex talibus
    in mundo: notum est, quod unaquaevis gens aliquod commune simile ferat in faciebus et oculis, per quod noscitur, et
    internoscitur ab alia gente; et adhuc magis una familia ab altera; sed hoc multo perfectius in caelis, quia ibi
    omnes affectiones interiores apparent et elucent ex facie, nam facies ibi est illarum forma externa et
    repraesentativa; aliam faciem habere quam suarum affectionum, non datur in caelo.
    Ostensum etiam est, quomodo communis similitudo variatur particulariter in singulis qui in una societate sunt: erat
    facies sicut angelica, quae mihi apparebat, et haec variabatur secundum affectiones boni et veri, quales sunt apud
    illos qui in una societate; variationes illae persistebant diu; et observabam, quod usque eadem facies in communi
    sicut planum permaneret, et quod reliquae essent modo derivationes et propagationes inde: sic etiam per hanc faciem
    ostensae sunt affectiones totius societatis, per quas variantur facies illorum qui ibi; nani, ut supra dictum est,
    facies angelicae sunt formae interiorum suorum, ita affectionum quae amoris et fidei.
\end{topic}

\begin{topic}{48}
    Inde etiam fit, quod angelus qui praestans sapientia est, videat illico ex facie qualis alter est; non potest
    quisquam ibi vultu recondere interiora, et simulare, et prorsus non mentiri et fallere astu et hypocrisi.
    Contingit aliquoties, quod in societates se insinuent hypocritae, qui edocti sunt recondere interiora sua, et
    componere exteriora ut appareant in forma boni, in quo sunt qui in societate, et sic mentiri lucis angelos; sed hi
    non diu ibi morari possunt, incipiunt enim angi interius, cruciari, livescere facie, et quasi exanimari: alterantur
    ita ex contrarietate vitae quae influit et operatur; quare se dejiciunt repente in infernum ubi similes, nec hiscunt
    amplius ascendere.
    Sunt illi qui intelliguntur per eum, qui inventus est inter discumbentes et invitatos, non indutus veste nuptiali,
    et ejectus in tenebras exteriores (\emph{Matth.} xxii. 11, seq.).
\end{topic}

\begin{topic}{49}
    Communicant omnes societates caeli inter se, non per apertum commercium, pauci enim exeunt e societate sua in aliam,
    nam exire e societate est sicut exire a se seu a sua vita, et transire in aliam quae non ita convenit; sed
    communicant omnes per extensionem sphaerae, quae procedit ex vita cujusvis: sphaera vitae est sphaera affectionum
    quae amoris et fidei; haec se extendit in societates circumcirca in longum et in latum, et eo longius et latius, quo
    affectiones sunt interiores et perfectiores.\footnote{Quod sphaera spiritualis, quae est sphaera vitae, effluat ex
    unoquovis homine, spiritu et angelo, et circumstipet illos (n. 4464, 5179, 7454, 8630).

    Quod effluat ex vita affectionis et cogitationis eorum (n. 2489, 4464, 6206).

    Quod sphaerae illae se longe extendant in societates angelicas secundum quale et quantum boni (n. 6598-6613[? 6612],
    8063, 8794, 8797).}
    Secundum extensionem illam est angelis intelligentia et sapientia: qui in intimo caelo sunt, et ibi in medio, habent
    extensionem in universum caelum; inde communicatio omnium caeli est cum unoquovis, et uniuscujusvis cum
    omnibus.\footnote{Quod in caelis detur communicatio omnium bonorum, quoniam amor caelestis communicat omnia sua cum
    altero (n. 549, 550, 1390, 1391, 1399, 10130, 10723).}
    Sed de hac extensione infra plenius agendum est, ubi de Forma cadesti, secundum quam angelicae Societates dispositae
    sunt; et quoque ubi de Sapientia et Intelligentia angelorum; nam omnis extensio affectionum et cogitationum vadit
    secundum illam formam.
\end{topic}

\begin{topic}{50}
    Dictum supra est, quod in caelis sint societates majores et minores; majores consistunt ex myriadibus, minores ex
    aliquot millibus, et minimae ex aliquot centenis angelis.
    Sunt etiam qui solitarii habitant, quasi domus et domus, familia et familia; hi tametsi ita dispersi vivunt, usque
    similiter ordinati sunt, sicut illi qui in societatibus, quod nempe sapientiores illorum in medio sint, et
    simpliciores in terminis: hi propius sub auspicio Divino Domini sunt, et sunt angelorum optimi.
\end{topic}
