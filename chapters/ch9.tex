% pdf page 51
\caput{Quod Unaquaevis Societas in Caelis Referat Unum Hominem.}

\begin{topic}{68}
    Quod unaquaevis societas caeli etiam referat unum hominem, et quoque in similitudine hominis sit, aliquoties mihi
    videre datum est.
    Erat societas, in quam se insinuaverunt plures, qui noverunt mentiri lucis angelos; erant hypocritae.
    Cum hi separarentur ab angelis, vidi quod integra societas primum appareret sicut unum obscurum, dein per gradus in
    forma humana etiam obscure, et tandem in luce sicut homo.
    Illi qui in nomine erant, et componebant illum, fuerunt qui in bono illius societatis erant; ceteri qui non in eo
    nomine erant, et non componebant illum, erant hypocritae: hi rejecti sunt, illi detenti; ita fiebat separatio.
    Hypocritae sunt, qui loquuntur bene, et quoque faciunt bene, sed spectant se in singulis: loquuntur sicut angeli de
    Domino, de caelo, de amore, de vita caelesti; et quoque faciunt bene, ut appareant quod tales sint sicut loquuntur:
    sed cogitant aliter, nihil credunt, nec volunt alicui bonum quam sibi; quod benefaciant, est propter se; si propter
    alios, est ut videantur, et sic quoque propter se.
\end{topic}

\begin{topic}{69}
    Quod integra societas angelica, cum Dominus Se praesentem sistit, appareat ut unum in forma humana, etiam datum est
    videre.
    Apparebat in alto versus ortum sicut nubes a candido rubescens cum stellulis circumcirca, quae descendebat; illa per
    gradus sicut descendit, lucidior facta est, et tandem visa in forma perfecte humana: stellulae circumcirca nubem
    erant angeli, qui ita apparuerunt a luce ex Domino.
\end{topic}

\begin{topic}{70}
    Sciendum est, quod tametsi omnes qui in una societate caeli sunt, quando simul ut unum apparent in similitudine
    hominis, usque non una societas sit similis homo sicut altera; distinguuntur inter se sicut facies humanae ex una
    stirpe; ex simili causa, de qua prius (n. 47), quod nempe varientur secundum varietates boni, in quo sunt, et quod
    format illos.
    In perfectissima et pulcherrima forma humana apparent societates quae in intimo seu supremo caelo sunt, et ibi in
    medio.
\end{topic}

\begin{topic}{71}
    Memoratu dignum est, quod quo plures in una societate caeli sunt, et illi unum agunt, eo ejus forma humana
    perfectior sit; nam varietas in formam caelestem disposita facit perfectionem, ut prius (n. 56) ostensum est; et
    varietas datur ubi plures.
    Omnis etiam societas caeli crescit numero indies; et sicut crescit, perfectior fit: sic non modo societas
    perficitur, sed etiam caelum in communi, quia societates constituunt caelum.
    Quoniam caelum ex crescente multitudine perficitur, patet quantum falluntur illi: qui credunt quod caelum claudatur
    ex plenitudine; cum tamen contrariami est, quod nusquam claudatur, et quod plenitudo major et major perficiat illud:
    quapropter angeli nihil potius desiderant, quam ut ad illos novi hospites angeli veniant.
\end{topic}

\begin{topic}{72}
    Quod unaquaevis societas sit in effigie hominis cum simul ut unum apparet, est quia totum caelum illam effigiem
    habet, ut in praecedente articulo ostensum videatur; et in perfectissima forma, qualis est forma caeli, similitudo
    est partium cum toto, et minorum cum maximo; minora et partes caeli sunt societates ex quibus consistit, quae quod
    etiam sint caeli in minore forma, videatur supra (n. 51-58).
    Quod perpetua talis similitudo sit, est quia in caelis omnium bona ex uno amore sunt, ita ex una origine.
    Unus amor ex quo origo omnium bonorum ibi, est amor in Dominum a Domino.
    Inde est, quod totum caelum sit similitudo Ipsius in communi, unaquaevis societas in minus communi, et unusquisque
    angelus in particulari.
    Videantur etiam quae supra (n. 58) de hac re dicta sunt.
\end{topic}
