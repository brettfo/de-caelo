% pdf page 22
\chapter{Quod Divinum Domini Faciat Caelum.}

\begin{topic}{7}
    Angeli simul sumpti dicuntur caelum, quia constituunt illud; sed usque est Divinum procedens a Domino, quod influit
    apud angelos, et quod recipitur ab illis, quod facit caelum in communi et in parte.
    Divinum procedens a Domino est bonum amoris et verum fidei; quantum itaque boni et veri recipiunt a Domino, tantum
    angeli sunt, et tantum caelum sunt.
\end{topic}

\begin{topic}{8}
    Unusquisque in caelis scit et credit, immo percipit, quod nihil boni ex se velit et faciat, et nihil veri ex se
    cogitet et credat, sed ex Divino, ita ex Domino; et quod bonum et verum quae a semet non sint bonum et verum, quia
    eis non vita a Divino inest.
    Angeli intimi caeli etiam clare percipiunt, et sentiunt influxum, et quantum recipiunt tantum videntur sibi in caelo
    esse, quia tantum in amore et fide, et tantum in luce intelligentiae et sapientiae, et in gaudio caelesti inde:
    quoniam omnia illa procedunt a Divino Domini, et in illis est caelum angelis, patet quod Divinum Domini faciat
    caelum, et non angeli ex aliquo proprio suo.
    \footnote{
        Quod angeli caeli agnoscant omne bonum esse a Domino, et nihil a semet; et quod Dominus in Suo habitet apud
        illos, et non in proprio illorum (n. 9338, 10125, 10151, 10157).

        Quod ideo in Verbo per angelos intelligatur aliquid Domini (n. 1925, 2821, 3039, 4085, 8192, 10528).

        Et quod ideo angeli dicantur dii a receptione Divini a Domino (n. 4295, 4402, 7268, 7873, 8301, 8192).

        Quod etiam a Domino sit omne bonum quod bonum, ac omne verum quod verum, proinde omnis pax, amor, charitas et
        fides (n. 1614, 2016, 2751, 2882, 2883, 2891, 2892, 2904).

        Et quod omnis sapientia et intelligentia (n. 109, 112, 121, 124).
    }
    Inde est quod caelum in Verbo dicatur ``Habitaculum Domini,'' ac ``Thronus Ipsius;'' et quod illi qui ibi dicantur
    in Domino esse.
    \footnote{
        Quod qui in caelo sunt dicantur esse in Domino (n. 3637, 3638).
    }
    Quomodo autem Divinum procedit a Domino, ac implet caelum, in sequentibus dicetur.
\end{topic}

\begin{topic}{9}
    Angeli ex sapientia sua adhuc ulterius progrediuntur; dicunt non modo quod omne bonum at verum sint a Domino, sed
    etiam omne vitae: confirmant id per hoc, quod nihil existere possit a se, sed a priori se, ita quod omnia existant a
    Primo, quod vocant ipsum Esse vitae omnium, et quod similiter subsistant, quoniam subsistere est perpetuo existere,
    et quod non in nexu continue tenetur per intermedia cum Primo, hoc illico dilabitur et prorsus dissipatur: aiunt
    insuper, quod modo unicus vitae fons sit, et quod vita hominis sit rivus inde, qui si non a fonte suo continue
    subsistit, quod illico duffluat.
    Porro quod ab unico illo fonte vitae, qui est Dominus, non procedat nisi quam Divinum bonum ac Divinum verum, et
    quod haec afficiant unumquemvis secundum receptionem; qui recipiunt illa fide et vita, quod in illis caelum sit; sed
    qui rejiciunt illa, vel suffocant illa, quod vertant illa in infernum, bonum enim vertunt in malum, et verum in
    falsum, ita vitam in mortem.
    Quod omne vitae a Domino sit, etiam confirmant per id, quod omnia in universo se referant ad bonum et verum, vita
    voluntatis hominis quae est via amoris ejus ad bonum, et vita intellectus hominis quae est vita fidei ejus ad verum;
    quare cum omne bonum et verum desuper venit, sequitur quod etiam omne vitae.
    Quia angeli ita credunt, ideo renuunt omnem gratiarum actionem propter bonum quod faciunt, ac indignantur et
    recedunt, si quis bonum illis tribuit: mirantur quod aliquis credat, quod sapiat ex se, et quod bonum faciat ex se:
    bonum facere propter se, hoc non vocant bonum, quid fit ex se; at bonum facere propter bonum, hoc vocant bonum ex
    Divino, et quod hoc bonum sit quod facit caelum, quia id bonum est Dominus.
    \footnote{
        Quod bonum a Domino intus in se habeat Dominum, non autem bonum a proprio (n. 1802, 3951, 8478[, 8480]).
    }
\end{topic}

\begin{topic}{10}
    Spiritus, qui dum in mundo vixerunt, in illa fide se confirmaverunt, quod bonum quod faciunt et verum quod credunt,
    sint a semet, aut appropriata sibi ut sua, in qua fide sunt omnes illi qui meritum ponunt in beneactis, ac justitiam
    sibi vindicant, illi non recipiuntur in caelum; angeli illos fugiunt, spectant illos ut stupidos ac ut fures, ut
    stupidos quia jugiter spectant ad se et non ad Divinum, ut fures quia auferunt Domino quod Ipsius est.
    Hi contra fidem caeli sunt, quod Divinum Domini apud angelos faciat caelum.
\end{topic}

\begin{topic}{11}
    Quod illi in Domino sint, et Dominus in illis qui in caelo et in ecclesia, docet quoque Dominus dicendo,
    \begin{quote}
        ``Manete in Me, et Ego in vobis; sicut palmes non potest ferre fructum a semetipso, nisi manserit in vite, ita
        nec vos nisi in Me manseritis: Ego sum Vitis, vos palmites; qui manet in Me, et Ego in illo, hic fert fructum
        multum; quia sine Me non potestis facere quicquam'' (\emph{Joh.} xv. 4-7).
    \end{quote}
\end{topic}

\begin{topic}{12}
    Ex his nunc constare potest, quod Dominus in Suo habitet apud angelos caeli, et sic quod Dominus sit omne in omnibus
    caeli; et hoc ex causa, quia bonum a Domino est Dominus apud illos, quod enim ab Ipso est Ipse est; proinde quod
    bonum a Domino sit angelis caelum, et non aliquod proprium illorum.
\end{topic}
