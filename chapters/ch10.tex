% pdf page 53
\caput{Quod Inde Unusquisque Angelus Sit in Perfecta Forma Humana.}

\begin{topic}{73}
    In binis praecedentibus articulis ostensum est, quod caelum in toto complexu referat unum hominem, et quod similiter
    unaquaevis societas in caelo: ex nexu causarum, quae ibi adductae sunt, sequitur quod unusquisque angelus pariter
    referat.
    Sicut caelum est homo in maxima forma, et societas caeli in minore, ita angelus est in minima; nam in perfectissima
    forma, qualis est forma caeli, similitudo totius est in parte, et partis in toto: causa quod ita sit, est, quia
    caelum est communio, communicat enim omnia sua cum unoquovis, et unusquisque recipit ex communione illa omnia sua;
    angelus est receptaculum, et inde caelum in minima forma, ut quoque in suo articulo supra ostensum est.
    Homo etiam, quantum reclpit caelum, tantum quoque est receptaculum, est caelum, et est angelus (videatur supra, n.
    57).
    Hoc describitur ita in \emph{Apocalypsi},
    \begin{quote}
        ``Mensus est murum'' sanictae Hierosolymae, ``centum quadraginta quatuor cubitorum, mensura hominis quae est
        angeli'' (xxi. 17):
    \end{quote}
    ``Hierosolyma'' ibi est ecclesia Domini, et in eminentiori sensu caelum;\footnote{``Hierosolyma'' quod sit ecclesia
    (n. 402, 3654, 9166).} ``murus'' est verum quod ab insultu falsorum et malorum tutatur;\footnote{``Murus'' quod sit
    verum tutans ab insultu falsorum et malorum (n. 6419).} ``centum quadraginta quatuor'' sunt omnia vera et bona in
    complexu;\footnote{Quod ``duodecim'' sint omnia vera et bona in complexu (n. 577, 2089, 2129, 2130, 3272, 3858,
    3913).

    Similiter ``septuaginta duo,'' et ``centum quadraginta quatuor,'' quoniam 144 exsurgunt ex 12 in se multiplicatis
    (n. 7973).

    Quod omnes numeri in Verbo significent res (n. 482, 487, 647, 648, 755, 813, 1963, 1988, 2075, 2252, 3252, 4264,
    4495, 5265).

    Quod numeri multiplicati simile significent cum simplicibus, a quibus per multiplicationem exsurgunt (n. 5291, 5335,
    5708, 7973).} ``mensura'' est quale ejus;\footnote{Quod ``mensura'' in Verbo significet quale rei quoad verum et
    bonum (n. 3104, 9603).} ``homo'' est in quo omnia illa in communi et in parte, ita in quo caelum; et quia angelus
    etiam est homo ex illis, ideo dicitur, ``mensura hominis quae est angeli.''
    Hic sensus spiritualis est illorum verborum.
    Quis absque illo sensu intellecturus esset, quod murus sanctae Hierosolymae esset mensura hominis quae
    angeli?\footnote{De sensu spirituali seu interno Verbi, vide Explicationem \emph{de Equo Albo}, in
    \emph{Apocalypsi}, et \emph{Appendicem ad Doctrinam Caelestem}.}
\end{topic}

\begin{topic}{74}
    Sed nunc ad experientiam.
    Quod angeli sint formae humanae seu homines, hoc millies mihi visum est: locutus enim sum cum illis sicut homo cum
    homine, quandoque cum uno, quandoque cum pluribus in consortio, nec quicquam differens ab homine quoad formam apud
    illos vidi; et miratus sum aliquoties quod tales essent: et ne diceretur quod esset fallacia aut visio phantasiae,
    datum est illos videre in plena vigilia, seu cum eram in omni sensu corporis, et in statu clarae perceptionis.
    Saepius etiam narravi illis, quod homines in Christiano orbe in caeca tali ignorantia sint de angelis et spiritibus,
    ut credant illos esse mentes absque forma, ac puras cogitationes, de quibus non aliam ideam habent quam sicut de
    aethereo in quo vitale; et quia sic addicunt illis nihil hominis praeter cogitativum, credunt quod non videant quia
    non eis oculi, non audiant quia non eis aures, et non loquantur quia non eis os et lingua.
    Ad haec angeli dixerunt, quod sciant quod talis fides sit multis in mundo, et quod regnet apud eruditos, et quoque,
    quod mirati sunt, apud sacerdotes.
    Causam etiam dixerunt, quod eruditi, qui antesignani fuerunt, et primum excluserunt talem ideam de angelis et
    spiritibus, ex sensualibus externi hominis de illis cogitaverint; et qui ex illis cogitant, et non ex luce
    interiore, et ex idea communi quae insita cuivis, non possint aliter quam fingere talia, quoniam sensualia externi
    hominis non capiunt alia quam quae intra naturam sunt, non autem quae supra, ita nihil quicquam de spirituali
    mundo:\footnote{Quod homo nisi a sensualibus externi hominis elevetur, parum sapiat (n. 5089).

    Quod sapiens homo supra sensualia illa cogitet (n. 5089, 5094).

    Cum elevatur homo supra sensualia illa, quod in lumen clarius veniat, et tandem in lucem caelestem (n. 6183, 6313,
    6315, 9407, 9730, 9922).

    Quod elevatio et abdudtio a sensualibus illis antiquis nota fuerit (n. 6313).} ex his antesignanis ut ducibus
    derivata est falsitas cogitationis de angelis ad alios, qui ex se non cogitaverunt sed ex illis; et qui ex aliis
    primum cogitant, et faciunt suae fidei, et postea illa suo intellecìu intuentur, aegre possunt ab illis recedere;
    quare plerique acquiescunt in confirmando illa.
    Porro dixerunt, quod simplices fide et corde non in illa idea de angelis sint, sed in idea de illis sicut de
    hominibus caeli, ex causa quia non exstinxerunt insitum suum quod e caelo per eruditionem, nec capiunt aliquid
    absque forma: inde est, quod angeli in templis, sive sculpti sive picti, non aliter sistantur quam ut homines.
    De insito quod e caelo dicebant, quod sit Divinum influens apud illos qui in bono fidei et vitae sunt.
\end{topic}

\begin{topic}{75}
    Ab omni experientia, quae nunc est plurium annorum, dicere et asseverare possum, quod angeli quoad formam suam sint
    prorsus homines, quod illis sint facies, sint oculi, aures, pectus, brachia, manus, pedes; quod se mutuo videant,
    audiant, loquantur inter se; verbo, quod illis prorsus nihil desit, quod est hominis, praeter quod non superinduti
    sint materiali corpore.
    Vidi illos in sua luce, quae lucem meridianam mundi multis gradibus excedit, et in illa omnia faciei illorum
    distinctius et clarius, quam visae sunt facies hominum telluris.
    Datum etiam est videre angelum intimi caeli: is nitentiori et splendentiore facie erat quam angeli inferiorum
    caelorum; lustravi eum, et erat ei forma humana in omni perfectione.
\end{topic}

\begin{topic}{76}
    At sciendum est, quod angeli non possint videri homini per oculos corporis ejus, sed per oculos spiritus qui est in
    nomine,\footnote{Quod homo quoad interiora sua sit spiritus (n. 1594).

    Et quod spiritus ille sit ipse homo, et quod ex illo corpus vivat (n. 447, 4622, 6054).} quia is est in spirituali
    mundo, et omnia corporis in naturali; simile videt simile, quia ex simili: praeterea organum visus corporis, quod
    est oculus, tam crassum est, ut ne quidem videat minora naturae nisi per vitra optica, ut cuivis notum est; inde
    minus adhuc illa quae supra naturae sphaeram sunt, qualia sunt omnia quae in spirituali mundo: sed haec usque
    videntur ab homine, cum is abducitur a visu corporis, et aperitur visus spiritus ejus, quod etiam momento fit, cum
    placet Domino ut videantur; et tunc homo non aliud scit quam quod videat illa oculis corporis: ita angeli visi sunt
    Abrahamo, Loto, Manoacho, et Prophetis; ita quoque visus est Dominus post resurrectionem discipulis: simili modo
    etiam mihi visi sunt angeli.
    Quia prophetae ita viderunt, ideo dicti sunt ``videntes'' et ``aperti oculis'' (1 \emph{Sam.} ix. 9; \emph{Num.}
    xxiv. 3); ac facere ut ita videant, dictum est ``aperire oculos,'' ut factum est puero Elisaei, de quo ita legitur,
    \begin{quote}
        ``Orans Elisaeus dixit, Jehovah, aperi quaeso oculos ejus ut videat; et aperiente Jehovah oculos pueri ejus,
        vidit quod ecce mons ille plenus equis et curribus igneis circa Elisaeum'' (2 \emph{Reg.} vi. 17).
    \end{quote}
\end{topic}

\begin{topic}{77}
    Spiritus probi, cum quibus de hac re etiam locutus sum, doluerunt corde, quod talis ignorantia de statu caeli, et de
    spiritibus et angelis intra ecclesiam esset; et indignati dicebant, quod omnino referrem quod non sint mentes absque
    forma, nec pneumata aetherea, sed quod sint homines in effigie, et quod videant, audiant, et sentiant aeque ac illi:
    qui in mundo.\footnote{Quod unusquisque angelus, quia est recipiens Divini ordinis a Domino, sit in humana forma
    perfecta et pulcra secundum receptionem (n. 322, 1880, 1881, 3633, 3804, 4622, 4735, 4797, 4985, 5199, 5530, 6054,
    9879, 10177, 10594).

    Quod Divinum verum sit per quod ordo, et Divinum bonum sit essentiale ordinis (n. 2451, 3166, 4390, 4409, 5232,
    7256, 10122, 10555).}
\end{topic}
