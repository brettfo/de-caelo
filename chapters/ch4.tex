% pdf page 29
\caput{Quod Caelum Distinctum sit in Duo Regna.}

\begin{topic}{20}
    Quoniam in caelo sunt infinitae varietates, et una societas no prorsus similis alteri, ne quidem unus angelus
    alteri,\footnote{Quod inifita varietas sit, et nusquam aliquid idem cum altero (n. 7236, 9002).} ideo distinguitur
    caelum in communi, in specie et in parte; in communi in duo regna, in specie in tres caelos, ac in parte in
    innumeras societates: de singulis in nunc sequentibus dicetur.
    Reegna dicuntur, quia caelum vocatur ``Regnum Dei.''
\end{topic}

\begin{topic}{21}
    Sunt angeli qui interius magis et minus recipiunt Divinum a Domino procedens; qui magis interius recipiunt, vocantur
    angeli caelestes, at qui minus interius vocantur angeli sprirituales: inde caelum distinguitur in duo regna, quorum
    unum vocatur \emph{Regnum Caeleste;} alterum \emph{Regnum Spirituale.}\footnote{Quod caelum in toto in duo regna
    distinctum sit, in regnum caeleste et in regnum sprituale (n. 3887, 4138).

    Quod angeli regni caelstis recipiant Divinum Domini in parte voluntaria, ita interius quam angeli sprituales, qui
    recipiunt illud in parte intellectuali (n. 5113, 6367, 8521, 9935[? 9936], 9995, 10124).}
\end{topic}

\begin{topic}{22}
    Angeli qui regnum caeleste constituunt, quia interius magis recipiunt Divinum Domini, vocantur angeli interiores et
    quoque superiores; ac inde etiam caeli, quos constituunt, vocantur caeli interiore ac superiores.\footnote{Quod
    caeli, qui constituunt regnum caeleste, dicantur superiores; qui autem regnum spirituale, inferiores (n. 10068).}
    Quod superiores et inferiores dicantur, est quia interiora et exteriora ita vocantur.\footnote{Quod interiora
    experimantur per superiora, et quod superiora significent interiora (n. 2148, 3084, 4599, 5146, 8325).}
\end{topic}

\begin{topic}{23}
    Amor in quo sunt qui in regno caelesti, vocatur amor caelistis; et amor in quo sunt qui in regno spirituali, vocatur
    amor spiritualis: amor caelestis est amor in Dominum, et amor spirituali est charitas erga proximum.
    Et quia omne bonum est amoris, nam quod aliquis amat, hoc ei bonum est, ideo etiam bonum unius regni vocatur
    caeleste, et alterius bonum spirituale.
    Inde patet, in quo se distinguunt bina illa regna, quod nempe sicut bonum amoris in Dominum, et bonum charitatis
    erga proximum:\footnote{Quod bonum regni caelestis sit bonum amoris in Dominum, et bonum rengi spiritualis sit bonum
    charitatis erga proximum (n. 3691, 6435, 9468, 9680, 9683, 9780).} et quia illud bonum est interius bonum, et ille
    amor est interior amor, ideo angeli caelestes sunt angeli interiores, et vocantur superiores.
\end{topic}

\begin{topic}{24}
    Regnum caeleste etiam vocatur regnum sacerdotale Domini, et in Verbo ``habitaculum Ipsius,'' et regnum sprituale
    vocatur regnum regium Ipsius, et in Verbo ``thronus Ipsius:'' ex Divino caelesti etiam Dominus in mundo appellatus
    est ``Jesus,'' et ex Divino sprituali ``Christus.''
\end{topic}

\begin{topic}{25}
    Angeli in regno caelesti Domini valde excellent sapientia et gloria prae angelis qui in regno spirituali, ex causa
    quia interius recipiunt Divinum Domini, sunt enim in amore in Ipsum, et inde Ipsi propiores et
    conjunctiores.\footnote{Quod angeli caelestes immensum sapiant prae angelis spiritualibus (n. 2718, 9995).

    Quale discrimen inter angelos caelestes et inter angelos spirituales (n. 2088, 2669, 2708, 2715, 3235, 3240, 4788,
    7068, 8121[? 8521], 9277, 10295).}
    Quod illi angeli tales sint, est quia receperunt et recipiunt Divina vera statim in vita, et non ut spirituales
    praevia memoria et cogitatione; quapropter habent illa inscripta cordibus suis, ac percipiunt illa et quasi vident
    illa in se, nec usquam ratiocinantur de illis num ita sit vel non ita:\footnote{Quod angeli caelestes non
    ratiocinentur de veris fidei, quia percipiunt illa in se, sed quod angeli spirituales ratiocinantur de illis num ita
    sit vel non ita (n. 202, 337, 597, 607, 784, 1121, 1387[? 1384], 1398[? 1385, 1394], 1919, 3246, 4448, 7680, 7877,
    8780, 9277, 10786).} sunt quales describuntur apud \emph{Jeremiam,}
    \begin{quote}
        ``Indam legem meam menti eorum, et cordi eorum inscribam eam:...non docebunt amplius quisquam amicum suum et
        quisquam fratrem suum, dicendo, Cognoscite Jehovam;..cognoscent Me a minimo eorum ad maximum eorum'' (xxxi. 33,
        34).
    \end{quote}
    Et vocantur apud \emph{Esaiam,}
    \begin{quote}
        ``Docti a Jehovah'' (liv. 13):
    \end{quote}
    quod qui docti a Jehovah sint qui docti a Domino, docet Ipse Dominus apud \emph{Johannem} (cap. vi. 45, 46).
\end{topic}

\begin{topic}{26}
    Dictum est, quod illis sapientia et gloria sit prae reliquis, quia receperunt et recipiunt Divina vera statim in
    vita; ut primum enim audiunt illa, etiam volunt et faciunt illa, nec reponunt in memoria, et dein cogitant num ita
    sit.
    Quid tales sunt, sciunt illico per influxum a Domino, num verum sit verum quod audiunt, influit enim Dominus
    immediate in velle hominis, et mediate per velle in ejus cogitare; seu quod idem, influit Dominus immediate in bonum
    ac mediate per bonum in verum;\footnote{Quod inflexus Domini sit in bonum et per bonum in verum et non vicissim; ita
    in voluntatem et peream intellectum, et non vicissim (n. 5482, 5649, 6027, 8685, 8701, 10153).} nam id bonum dicitur
    quod est voluntatis et inde operis, at verum quod est memoriae et inde cogitationis: etiam omne verum vertitur in
    bonum ac implantatur amori, ut primum intrat voluntatem; quamdiu autem verum est in memoria et inde cogitatione, non
    fit bonum, nec vivit, nec appropriatur homini, quonima homo est homo ex vountate et inde intellectu, et non ex
    intellectu separato a voluntate.\footnote{Quod voluntas hominis sit ipsum esse vitae illius, et quod sit
    receptaculum boni amoris, et quod intellectus sit existere vitae inde, et quod sit receptaculum veri et boni fidei
    (n. 3619, 5002, 9282).

    Ita quod vita voluntatis sit vita principalis hominis, et quod vita intellectus procedat inde (n. 585, 590, 3619,
    7342, 8885, 9285[? 9282], 10076, 10109, 10110).

    Quod illa fiant vitae, et approprientur homini, quae recipiuntur voluntate (n. 3161. 9386, 9393).

    Quod homo sit homo ex voluntate et inde intellectu (n. 8911, 9069, 9071, 10076, 10106, 10110).

    Unusquisque etiam ab aliis amatur et aestimatur qui vult bene et intelligit bene, ac rejicitur et vilipenditur qui
    intelligit bene et non vult bene (n. 8911, 10076).

    Quod homo etiam post mortem maneat sicut ejus voluntas et inde ejus intellectus, et quod quae intellectus sunt et
    non simul voluntatis tunc evanescant, quia non sunt in homine (n. 9069. 9071, 9282, 9386, 10153).}
\end{topic}

\begin{topic}{27}
    Quia tale discrimen est inter angelos regni caelestis et inter angelos regni spiritualis, ideo non simul sunt nec
    consortium inter se habent; datur modo communicatio per societates angelicas intermedias, quae vocantur caelestes
    spirituales; per has influit regnum caeleste in sprituale:\footnote{Quod inter duo regna sit communicatio et
    conjunctio per societates angelicas, quae vocantur caelestes spirituales (n. 4047, 6435, 8787[? 8796], 8881[?
    8802]).
    De influxu Domini per regnum caeleste in spirituale (n. 3969, 6366).} inde fit, quod tametsi caelum in duo regna
    divisum est, usque unum faciat.
    Dominus semper providet angelos tales intermedios, per quos communicatio et conjunctio.
\end{topic}

\begin{topic}{28}
    Quia multis agitur in sequentibus de angelis unius et alterius regni, ideo specifica hic praetereuntur.
\end{topic}
