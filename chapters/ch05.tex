% pdf page 33
\caput{Quod Tres Caeli Sint.}

\begin{topic}{29}
    Tres caeli sunt, et illi inter se distinctissimi, intimum seu tertium, medium seu secundum, et ultimum seu primum;
    consequuntur illi ac subsistunt inter se, sicut supremum hominis quod caput vocatur, medium ejus quod corpus, et
    ultimum quod pedes; et sicut suprema pars domus, media ejus, et infima ejus: in tali ordine etiam est Divinum quod a
    Domino procedit et descendit; inde ex necessitate ordinis caelum tripartitum est.
\end{topic}

\begin{topic}{30}
    Interiora hominis quae ejus mentis et animi sunt etiam in simili ordine sunt; est ei intimum, medium, ac ultimum,
    nam in hominem, dum creatus, omnia ordinis Divini collata sunt, adeo ut factus sit Divinus ordo in forma, et inde
    caelum in minima effigie:\footnote{Quod in hominem omnia Divini ordinis collata sint, et quod homo a creatione sit
    Divinis ordo in forma (n. 4219, 4220[? 4222], 4223, 4523, 4524, 5114, 5368[? 3628, 5168], 6013, 6057, 6605, 6626,
    9706, 10156, 10472).

    Quod apud hominem internus ejus homo formatus sit ad imaginem caeli, et externus ad imaginem mundi, et quod ideo
    homo ab antiquis dictus sit microcosmus (n. 4523, 5368[? 3628, 5115], 6013, 6057, 9279, 9706, 10156, 10472).

    Quod sic homo ex creatione quoad interiora sua sit caelum in minima effigie ad imaginem maximi, et quod etiam talis
    sit homo qui e novo creatus est seu regeneratus a Domino (n. 911, 1900, 1982[? 1928], 3624-3631, 3634, 3884, 4041,
    4279, 4523, 4524, 4625, 6013, 6057, 9279, 9632).} ideo etiam homo communicat cum caelis quoad interiora sua; et
    quoque inter angelos venit post mortem, inter angelos intimi caeli, aut medii, aut ultimi, secundum receptionem
    Divini boni et veri a Domino, cum vivit in mundo.
\end{topic}

\begin{topic}{31}
    Divinum quod influit a Domino et recipitur in tertio seu intimo caelo, vocatur caeleste, et inde angeli qui ibi
    vocantur angeli caelestes; Divinum quod influit a Domino et recipitur in secundo seu medio caelo, vocatur
    spirituale, et inde angeli qui ibi angeli spirituales; Divinum autem quod influit a Domino et recipitur in ultimo
    seu primo caelo, vocatur naturale; ast quia naturale illius caeli non est sicut naturale mundi, sed in se habet
    spirituale et caeleste naturale, et inde angeli qui ibi spirituales et caelestes naturales;\footnote{Quod tres caeli
    sint, intimum, medium, et ultimum, seu tertium, secundum, et primum (n. 684, 8594[? 9594], 10270).

    Quod etiam in triplici ordine sequantur bona ibi (n. 4938, 4939, 9992, 10005, 10017).

    Quod bonum intimi seu tertii caeli dicatur caeleste, bonum medii seu secundi spirituale, et bonum ultimi seu primi
    spirituale naturale (n. 4279, 4286, 4938, 4639, 9992, 10005, 10017, 10068).} spirituales naturales vocantur, qui
    influxum recipiunt ex caelo medio seu secundo, quod est caelum sprituale; ac caelestes naturales vocantur, qui
    influxum recipiunt ex caelo tertio seu intimo, quod est caelum caeleste: angeli spirituales naturales et caelestes
    naturales distincti sunt inter se, sed usque unum caelum constituunt, quia in uno gradu sunt.
\end{topic}

\begin{topic}{32}
    Est in unoquovis caelo internum et externum: qui in interno sunt, vocantur ibi angeli interni; quia autem in externo
    sunt, vocantur ibi angeli externi.
    Externum et internum in caelis, seu in unoquovis caelo, se habent sicut voluntarium et ejus intellectuale apud
    hominem; internum sicut voluntarium, et externum sicut ejus intellectuale: omne voluntarium habet suum
    intellectuale; unum absque altero non datur; se habet voluntarium sicut comparative flamma, et intellectuale ejus
    sicut lux unde.
\end{topic}

\begin{topic}{33}
    Probe sciendum est, quod interiora apud angelos faciant, ut sint in uno aut in altero caelo: quo enim interiora
    apertiora ad Dominum sunt, eo in interiori caelo sunt.
    Tres gradus interiorum sunt apud unumquemvis tam angelum quam spiritum, et quoque apud hominem; illi apud quos
    tertius gradus apartus est, in intimo caelo sunt; apud quos secundus, aut modo primus, illi in medio aut ultimo
    caelo sunt.
    Interiora aperiuntur per receptionem Divini boni ac Divini veri: qui afficiuntur Divinis veris, et admittunt illa
    statim in vitam, ita in voluntatem et ide actum, in intimo seu tertio caelo sunt, et ibi secundum receptionem boni
    ex affectione veri; qui autem non admittunt illa statim in voluntatem, sed in memoriam et inde intellectum, et ex eo
    volunt et faciunt ea, illi in medio seu secundo caelo sunt: at qui moraliter vivunt, et credunt Divinum, nec
    tantopere curant instrui, illi in ultimo seu primo caelo sunt.\footnote{Quod totidem gradus vitae in homine sint,
    quot caeli, et quod aperiantur post mortem secundum ejus vitam (n. 3747, 9594).

    Quod caelum sit in homine (n. 3884).

    Inde quod qui caelum in se recepit in mundo, in caelum veniat post mortem (n. 10717).}
    Inde constare poteest, quod status interiorum faciant caelum, et quod caelum sit intra unumquemvis, et non extra
    illum; quod etiam Dominus docet, dicendo,
    \begin{quote}
        ``Non venit regnum Dei cum observatione, neque dicent, Ecce hic aut ecce illic; ecce enim regnum Dei in vobis
        habetis'' (\emph{Luc.} xvii. 20, 21).
    \end{quote}
\end{topic}

\begin{topic}{34}
    Omnis etiam perfectio crescit versus interiora, et decrescit versus exteriora, quoniam interiora sunt propiora
    Divino et in se puriora, exteriora autem remotiora a Divino et in se crassiora.\footnote{Quod interiora sint
    perfectiora quia propiora Divino (n. 3405, 5146, 5147).

    Quod in interno millia et millia sint, quae in externo apparent ut commune unum (n. 5707).

    Quod quantum homo ab externis elevatur versus interiora, tantum in lucem et ita in intelligentiam veniat, et quod
    elevatio sit sicut e nimbo in clarum (n. 4598, 6183, 6333[? 6313]).}
    Perfectio angelica consistit in intelligentia, in sapientia, in amore, inque omni bono, et inde felicitate; non
    autem in felicitate absque illis, nam felicitas absque illis est externa et non interna.
    Quia interiora apud angelos intimi caeli aperta sunt in tertio gradu, ideo perfectio illorum immensum superat
    perfectionem angelorum in medio caelo, quorum interiora aperta sunt in secundo gradu; similiter excedit perfectio
    angelorum medii caeli perfectionem angelorum ultimi caeli.
\end{topic}

\begin{topic}{35}
    Quia tale discrimen est, non potest angelus unius caeli intrare ad angelos alterius caeli, seu non potest aliquis ex
    interiori caelo ascendere, nec aliquis ex superiori caelo descendere: qui ex inferiori caelo ascendit, corripitur
    anxietate usque ad dolorem, nec potest videre illos qui ibi, minus loqui cum illis; et qui ex superiori caelo
    descendit, privatur sua sapientia, titubat voce, et desperat.
    Fuerunt quidam ex ultimo caelo, qui nondum instructi erant quod caelum consisteret in interioribus angeli, credentes
    quod in superiorem felicitatem caelestem venirent, modo in caelum ubi illi angeli; permittebatur etiam ut ad illos
    intrarent, at cum ibi erant, neminem videbant utcunque inquirerent, tametsi magna multitudo erat; advenarum enim
    interiora non aperta erant in tali gradu in quo interiora angelorum qui ibi, inde nec visus; et paulo post
    corripiebantur angore cordis, usque adeo ut vix scirent num in vita essent vel non; quapropter subito inde se
    contulerunt ad caelum unde erant, gavisi quod inter suos venirent; spondentes quod non amplius cuperent altiora,
    quam quae vitae eorum concordant.
    Vidi etiam demissos e caelo superiori, et privatos sua sapientia, usque ut nescirent quale esset suum caelum.
    Secus fit, cum Dominus elevat aliquos ex inferiori caelo in superius, ut videant gloriam ibi, quod fit saepius; tunc
    praeparantur primum, et stipantur angelis intermediis per quos communicatio.
    Ex his patet quod tres illi caeli inter se distinctissimi sint.
\end{topic}

\begin{topic}{36}
    Qui autem in eodem caelo sunt, illi consociari possunt cum quibuscunque ibi, at jucunda consociationis se habent
    secundum affinitates boni, in quibus sunt: sed de his in sequentibus articulis.
\end{topic}

\begin{topic}{37}
    Verum tametsi caeli ita distincti sunt, ut angeli unius caeli non sociare possint commercium cum angelis alterius,
    usque tamen Dominus conjungit omnes caelos per influxum immediatum et mediatum; per influxum immediatum ex Se in
    omnes caelos, et per mediatum ab uno caelo in alterum;\footnote{Quod influxus a Domino sit immediatus a Se, et
    quoque mediatus per unum caelum in alterum, et apud hominem similiter in interiora ejus (n. 6063, 6307, 6472, 9682,
    9683).

    De immediato influxu Divini a Domino (n. 6058, 6474-6478, 8717, 8728).

    De mediato influxu per mundum spiritualem in mundum naturalem (n. 4067, 6982, 6985, 6996).} et sic efficit, ut tres
    caeli unum sint, et omnes in nexu sint a Primo ad ultimum, usque adeo ut inconnexum non detur; quod non connexum est
    per intermedia cum Primo, hoc nec subsistit, sed dissipatur et fit nullum.\footnote{Quod omnia existant a prioribus
    se, ita a Primo, et quod similiter subsistant, quia subsistentia est perpetua existentia; et quod ideo inconnexum
    non detur (n. 3626, 3627, 3628, 3648, 4523, 4524, 6040, 6056).}
\end{topic}

\begin{topic}{38}
    Qui non scit quomodo se habet cum ordine Divino quoad gradus, non capere potest quomodo caeli distincti sunt, ne
    quidem quid internus et externus homo.
    Plerique in mundo non aliam notionem de interioribus et exterioribus seu de superioribus et inferioribus habent,
    quam sicut de continuo aut de cohaerente per continuum a puriori ad crassius; at interiora et exteriora se non
    habent continue, sed discrete.
    Sunt duplicis generis gradus; sunt gradus continui et sunt gradus non continui.
    Gradus continui se habent sicut gradus descrescentiae lucis a flamma usque ad suum obscurum; aut sicut gradus
    decrescentiae visus ab illis quae in luce sunt ad illa quae in umbra; aut sicut gradus puritatis atmosphaerae ab imo
    ad ejus summum; distantiae determinant hos gradus.
    At gradus non continui sed discreti, discriminati sunt sicut prius et posterius, sicut causa et effectus, et sicut
    producens et productum: qui explorat videbit, quod in omnibus et singulis in universo mundo, quaecunque sunt, tales
    gradus productionis et compositionis sint, quod nempe ab uno alterum et ab altero tertium, et sic porro.
    Qui non perceptionem horum graduum sibi comparat, nequaquam potest scire discrimina caelorum, et discrimina
    facultatum interiorum et exteriorum hominis, nec discrimen inter mundum spiritualem et mundum naturalem, nec
    discrimen inter spiritum hominis et corpus ejus; et inde nec intelligere potest quid et unde correspondentiae et
    repraesentationes, neque qualis est influxus; sensuales homines haec discrimina non capiunt, faciunt enim
    crescentias et decrescentias etiam secundum hos gradus continuas; inde non concipere possunt spirituale aliter quam
    sicut purius naturale: quapropter etiam foris stant, et e longinquo ab intelligentia.\footnote{Quod interiora et
    exteriora non continua sint, sed secundum gradus distincta et discreta, et quilibet gradus terminatus (n. 3691,
    4145[? 5145], 5114, 8603, 10099).

    Quod unum formatum sit ab altero, et quod quae sic formata sunt, non continue puriora et crassiora sint (n. 6326,
    6465).

    Qui non percipit distinctionem interiorum et exteriorum secundum tales gradus, quod non capere possit internum et
    externum hominem, nec caelos interiores et exteriores (n. 5146, 6465, 10099, 10181).}
\end{topic}

\begin{topic}{39}
    Ultimo licet arcanum quoddam de angelis trium caelorum memorare, quod prius non alicui in mentem venit, quia non
    intellexit gradus: quod nempe apud unumquemvis angelum, et quoque apud unumquemvis hominem, sit gradus intimus seu
    supremus, seu intimum et supremum quoddam, in quod Divinum Domini primum aut proxime influit, et ex quo disponit
    reliqua interiora quae secundum gradus ordinis apud illos succedunt; hoc intimum seu supremum vocari potest
    introitus Domini ad angelum et ad hominem, ac ipsissimum Ipsius domicilium apud illos.
    Per hoc intimum aut supremum homo est homo, et distinguitur a brutis animalibus, nam haec illud non habent; inde
    est, quod homo, secus ac animalia, possit quoad omnia interiora, quae sunt mentis et animi ejus, elevari a Domino ad
    Se, possit credere in Ipsum, affici amore in Ipsum, et sic videre Ipsum, et quod possit recipere intelligentiam et
    sapientiam, et loqui ex ratione; inde quoque est quod vivat in aeternum.
    Quid autem disponitur et providetur a Domino in eo intimo, non influit manifeste in perceptionem alicujus angeli,
    quia est supra ejus cogitationem, et excedit ejus sapientiam.
\end{topic}

\begin{topic}{40}
    Haec nunc sunt communia de tribus caelis; in sequentibus autem de unoquovis caelo in specie dicendum est.
\end{topic}
