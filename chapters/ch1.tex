% pdf page 19
\caput{Quod Dominus sit Deus Caeli.}

\begin{topic}{2}
    Primum erit scire quis Deus caeli est, quoniam reliqua inde pendent.
    In universo caelo non alius agnoscitur pro Deo caeli quam solus Dominus; dicunt ibi sicut Ipse docuit,
    \begin{quote}
        Quod unus sit cum Patre; quod Pater in Ipso, et Ipse in Patre; et quod qui videt Ipsum, videat patrem: et quod
        omne sanctum ab Ipso procedat (\emph{Joh.} x. 30, 38; cap. xiv. [9,] 10, 11; cap. xvi. 13-15).
    \end{quote}
    Locutus sum cum angelis saepius de hac re, et constanter dixerunt, quod non possint in caelo dintinguere Divinum in
    tria, quoniam sciunt et percipiunt quod Divinum unum sit, et quod unum sit in Domino: dixerunt etiam, quod qui ab
    ecclesia ex mundo veniunt, apud quos idea trium Divinorum est, non possint admitti in caelum, quoniam errat eorum
    cogitatio ab uno ad alterum, et non ibi licet cogitare tres et dicere unum,
    \footnote{
        Quod Christiani in altera vita explorati, qualem ideam de Deo uno haberent, et quod compertum sit quod haberent
        ideam trium deorum (n. 2329, 5256, 10736, 10738, 10821).

        Quod Divinum Trinum in Domino agnoscatur in caelo (n. 14, 15, 1729, 2005, 5256, 9303).
    }
    quia quisque in caelo ex cogitatione loquitur, est enim ibi loquela cogitativa seu cogitatio loquens; quare qui in
    mundo distinxerunt Divinum in tria, ac separatam ideam de unoquovis acceperunt, et non illam in Domino unam
    fecerunt et concentraverunt, non recipi possunt; datur enim in caelo omnium cogitationum communicatio, quaere si
    illuc veniret, qui cogitat tres et dicit unum, statim internosceretur et rejiceretur.
    Sed sciendum est, quod omnes illi qui non separaverunt verum a bono seu fidem ab amore, in altera vita, cum
    instructi, recipiant ideam caelestem de Domino, quod sit Deus universi; aliter vero qui fidem separaverunt a vita,
    hoc est, qui non vixerunt secundum praecepta verae fidei.
\end{topic}

\begin{topic}{3}
    Qui intra ecclesiam negaverunt Dominum, et agnoverunt solum Patrem, et in tali fide se confirmaverunt, illi extra
    caelum sunt; et quia non datur apud illos aliquis influxus e caelo, ubi Dominus solus adoratur, privantur per gradus
    facultate cogitandi verum de quacunque re, et tandem fiunt vel sicut muti, vel loquuntur stolide, et in eundo errant
    ac brachia eorum pendent et vibrantur sicut expertia virium in internodiis.
    Qui autem negaverunt Divinum Domini, et agnoverunt solum Humanum Ipsius, ut Sociniani, illi similiter extra caelum
    sunt, ac feruntur antrorsum paulo versus dextrum, ac in profundum demittuntur, et sic prorsus separantur a reliquis
    e Christiano orbe.
    Qui autem dicunt se credere in Divinum invisibile, quid nominant Ens universi a quo omnia exstiterant, ac rejiciunt
    fidem de Domino, illi experti sunt quod in nullum Deum credant, quia Divinum invisible est illis quale est naturae
    in suis primis, in quod non cadit fides et amor, quia non cogitatio:
    \footnote{
        Quod Divinum non perceptibile aliqua idea, non receptibile sit fide (n. 4733, 5110, 5633[? 5663], 6982, 6996,
        7004, 7211, 9267[? 9356, 10267], 9359, 9972, 10067).
    }
    illi relegantur inter illos, qui vocantur naturalistae.
    Aliter fit cum illis qui extra ecclesiam nati sunt, qui Gentes vocantur; de quibus in sequentibus.
\end{topic}

\begin{topic}{4}
    Omnes infantes, ex quibus tertia pars caeli, initiantur in agnitionem et fidem, quod Dominus sit eorum Pater, et
    postea quod sit omnium Dominus, ita Deus cali et terrae.
    Quod infantes adolescant in caelis, et perficiantur per cognitiones, usque in angelicam intelligentiam et
    sapientiam, videbitur in sequentibus.
\end{topic}

\begin{topic}{5}
    Quod Dominus sit Deus caeli, non ambigere possunt illi qui ab ecclesia sunt; docuit enim Ipse,
    \begin{quote}
        Quod omnia Patris, Ipsius sint (\emph{Matth.} xi. 27; \emph{Joh.} xvi. 15; cap. xvii. 2); Et quod Ipsi imnis
        potestas sit in caelo et in terra (\emph{Matth.} xxviii. 18):
    \end{quote}
    ``in caelo et in terra'' dicit, quoniam qui caelum regit etiam terram regit, unum enim pendet ab altero.
    \footnote{
        Quod universum caelum Domini sit (n. 2751, 7086).
        Quod Ipsi potestas in caelis et in terris (n. 1607, 10089, 10827).
        Quod quia Dominus regit caelum etiam regat onmia quae inde pendent, ita omnia in mundo (n. 2026, 2027, 4523,
        4524).
        Quod Domino soli sit potestas removendi inferna, detinendi a malis, ac tenendi in bono, ita salvandi (n.
        10019).
    }
    Regere caelum et terram, est recipere ab Ispo omne bonum quod amoris, et omne verum quod fidei, ita omnem
    intelligentiam et sapientiam, et sic omnem felicitatem; in summa, vitam aeternam.
    Hoc etiam Dominus docuit, dicendo,
    \begin{quote}
        ``Qui credit in Filium, habet vitam aeternam; qui vero non credit Filio, non videbit vitam'' (\emph{Joh.} iii.
        36).
    \end{quote}
    Alibi,
    \begin{quote}
        ``Ego sum resurrectio et vita; qui credit in Me, etsi moritur, vivet; omnis qui vivit et credit in Me, non
        morietur in aeternum'' (\emph{Joh.} xi. 25, 26).
    \end{quote}
    Et alibi,
    \begin{quote}
        ``Ego sum via, veritas, et vita'' (\emph{Joh.} xiv. 6).
    \end{quote}
\end{topic}

\begin{topic}{6}
    Fuerunt quidam spiritus, qui, dum in mundo vixerunt, professi sunt Patrem, et de Domino non aliam ideam; quam sicut
    de alio homine habuerunt, et inde non crediderunt Ipsum esse Deum caeli; quapropter illis permittebatur
    circumvagari, et inquirere ubicunque vellent, num aliud caelum sit quam Domini; inquisiverunt etiam per aliquot
    dies, et nullibi invererunt.
    Erant illi inter tales, qui felicitatem caeli ponebant in gloria ac in dominatu; et quia non potiri potuerunt quae
    cupiverunt, et dictum illis quod caelum non consistat in talibus, indignati sunt, et voluerunt caelum habere in quo
    possent dominari supra alios, et eminere, gloria quali in mundo.
\end{topic}
